\documentclass[mathserif]{beamer}
\usetheme{Warsaw}
\usepackage{proof}

\title{Tableau Theorem Prover\\ for Intuitionistic Propositional Logic}
\author{Larry Diehl}
\institute{Portland State University}
\date[Fall 2014]
{CS 510 - Mathematical Logic and Programming Languages}

\def\imp{\supset}

\newcommand{\turn}[1]{\Gamma \vdash #1}
\newcommand{\cturn}[2]{\Gamma , #1 \vdash #2}

\begin{document}

\frame{\titlepage}

\begin{frame}
\frametitle{Motivation}
\framesubtitle{Tableau for Classical Logic}

\begin{itemize}
\item If $\neg A$ is contradictory in all paths, then $A \lor \neg A$
  lets us conclude $A$ is a {\it tautology}.
\item For {\it satisfaction}, running tableau on $A$ yield a
  (classical model) evaluation context $\sigma$.
\item Tableau seems awfully tied to classical logic, is intuitionistic
  tableau doomed!?
\end{itemize}



\end{frame}


\begin{frame}
\frametitle{Classical vs Intuitionistic Logic}

%% \begin{columns}[c] % the "c" option specifies center vertical alignment
%% \column{.35\textwidth}
%% Contents of the first column

%% \column{.35\textwidth}
%% Contents split \\ into two lines
%% \end{columns}

\begin{block}{Classical Logic}
\begin{itemize}
\item The {\it meaning} of a proposition is its truth value.
\item {\bf Satisfaction:} Does evaluating it yield true?
\item $A \lor \neg A$
\item $\neg\neg A \Rightarrow A$
\end{itemize}
\end{block}

\begin{block}{Intuitionistic Logic}
\begin{itemize}
\item The {\it meaning} of a proposition is its constructive content.
\item {\bf Satisfaction:} Can you write it as a program?
\item $\neg\neg A \nRightarrow A$
\end{itemize}
\end{block}

\end{frame}


\begin{frame}
\frametitle{Proof Theory for Intuitionistic Logic}

$$
\infer
  [\imp_I]
  {\turn{A \imp B}}
{
  {\cturn{A}{B}}
}
\qquad
\infer
  [\imp_E]
  {\turn{B}}
{
  {\turn{A \imp B}}
  &
  {\turn{A}}
}
$$

$$
\infer
  [\land_I]
  {\turn{A \land B}}
{
  {\turn{A}}
  &
  {\turn{B}}
}
\qquad
\infer
  [\land_{E_1}]
  {\turn{A}}
{
  {\turn{A \land B}}
}
\qquad
\infer
  [\land_{E_2}]
  {\turn{B}}
{
  {\turn{A \land B}}
}
$$

$$
\infer
  [\lor_{I_1}]
  {\turn{A \lor B}}
{
  {\turn{A}}
}
\qquad
\infer
  [\lor_{I_2}]
  {\turn{A \lor B}}
{
  {\turn{B}}
}
\qquad
\infer
  [\lor_E]
  {\turn{C}}
{
  {\cturn{A}{C}}
  &
  {\cturn{B}{C}}
}
$$

$$
\infer
  [\neg_I]
  {\turn{\neg A}}
{
  {\cturn{A}{\bot}}
}
\qquad
\infer
  [\neg_E]
  {\turn{\bot}}
{
  {\turn{A}}
  &
  {\turn{\neg A}}
}
$$

$$
\infer
  [\top_I]
  {\turn{\top}}
{
}
\qquad
\infer
  [\bot_E]
  {\turn{A}}
{
  {\turn{\bot}}
}
$$

\end{frame}

\begin{frame}
\frametitle{Proof Theory for Classical Logic}

\center{... intuitionistic rules plus:}

$$
\infer
  [\neg\neg_I]
  {\turn{\neg\neg A}}
{
  {\turn{A}}
}
\qquad
\infer
  [\neg\neg_E]
  {\turn{A}}
{
  {\turn{\neg\neg A}}
}
$$

\center{...or...}

$$
\infer{\turn{A \lor \neg A}}
{}
$$

\end{frame}

\begin{frame}
\frametitle{Reference Smullyan vs Heyting books}
\framesubtitle{A bit more information about this}

\end{frame}

\end{document}

