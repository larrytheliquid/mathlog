\documentclass[mathserif]{beamer}
\usetheme{Warsaw}
\usepackage{proof}

\title{Tableau Theorem Prover\\ for Intuitionistic Propositional Logic}
\author{Larry Diehl}
\institute{Portland State University}
\date[Fall 2014]
{CS 510 - Mathematical Logic and Programming Languages}

\def\|{~|~}
\def\T{\textrm{T}}
\def\F{\textrm{F}}
\def\true{\textrm{true}}
\def\false{\textrm{false}}
\def\imp{\supset}
\def\dfn{~\Leftrightarrow~}
\def\arr{\Rightarrow}

\newcommand{\turn}[1]{\Gamma \vdash #1}
\newcommand{\cturn}[2]{\Gamma , #1 \vdash #2}

\newcommand{\eval}[1]{\sigma \triangleright #1}
\newcommand{\sat}[1]{\sigma \vDash #1}
\newcommand{\nsat}[1]{\sigma \nvDash #1}
\newcommand{\force}[1]{\Gamma \Vdash #1}
\newcommand{\dforce}[1]{\Delta \Vdash #1}
\newcommand{\forcep}[1]{\Gamma \Vdash^p #1}
\newcommand{\nforce}[1]{\Gamma \nVdash #1}

\begin{document}

\frame{\titlepage}

\begin{frame}
\frametitle{Motivation}
\framesubtitle{Tableau for Classical Logic}

\begin{itemize}
\item If $\neg A$ is contradictory in all paths, then $A \lor \neg A$
  lets us conclude $A$ is a {\it tautology}.
\item For {\it satisfaction}, running tableau on $A$ yield a
  (classical model) evaluation context $\sigma$.
\item Tableau seems awfully tied to classical logic, is intuitionistic
  tableau doomed!?
\end{itemize}



\end{frame}


\begin{frame}
\frametitle{Classical vs Intuitionistic Logic}

%% \begin{columns}[c] % the "c" option specifies center vertical alignment
%% \column{.35\textwidth}
%% Contents of the first column

%% \column{.35\textwidth}
%% Contents split \\ into two lines
%% \end{columns}

\begin{block}{Classical Logic}
\begin{itemize}
\item The {\it meaning} of a proposition is its truth value.
\item {\bf Satisfaction:} Does evaluating it yield true?
\item $A \lor \neg A$
\item $\neg\neg A \imp A$
\item $A \imp \neg\neg A$
\end{itemize}
\end{block}

\begin{block}{Intuitionistic Logic}
\begin{itemize}
\item The {\it meaning} of a proposition is its constructive content.
\item {\bf Satisfaction:} Can you write it as a program?
\item $A \imp \neg\neg A$
\end{itemize}
\end{block}

\end{frame}


\begin{frame}
\frametitle{Proof Theory for Intuitionistic Logic}

$$
\infer
  [\imp_I]
  {\turn{A \imp B}}
{
  {\cturn{A}{B}}
}
\qquad
\infer
  [\imp_E]
  {\turn{B}}
{
  {\turn{A \imp B}}
  &
  {\turn{A}}
}
$$

$$
\infer
  [\land_I]
  {\turn{A \land B}}
{
  {\turn{A}}
  &
  {\turn{B}}
}
\qquad
\infer
  [\land_{E_1}]
  {\turn{A}}
{
  {\turn{A \land B}}
}
\qquad
\infer
  [\land_{E_2}]
  {\turn{B}}
{
  {\turn{A \land B}}
}
$$

$$
\infer
  [\lor_{I_1}]
  {\turn{A \lor B}}
{
  {\turn{A}}
}
\qquad
\infer
  [\lor_{I_2}]
  {\turn{A \lor B}}
{
  {\turn{B}}
}
\qquad
\infer
  [\lor_E]
  {\turn{C}}
{
  {\cturn{A}{C}}
  &
  {\cturn{B}{C}}
}
$$

$$
\infer
  [\neg_I]
  {\turn{\neg A}}
{
  {\cturn{A}{\bot}}
}
\qquad
\infer
  [\neg_E]
  {\turn{\bot}}
{
  {\turn{A}}
  &
  {\turn{\neg A}}
}
$$

$$
\infer
  [\top_I]
  {\turn{\top}}
{
}
\qquad
\infer
  [\bot_E]
  {\turn{A}}
{
  {\turn{\bot}}
}
$$

\end{frame}

\begin{frame}
\frametitle{Proof Theory for Classical Logic}

\center{... intuitionistic rules plus ...}

$$
\infer
  [\neg\neg_I]
  {\turn{\neg\neg A}}
{
  {\turn{A}}
}
\qquad
\infer
  [\neg\neg_E]
  {\turn{A}}
{
  {\turn{\neg\neg A}}
}
$$

\center{...or...}

$$
\infer{\turn{A \lor \neg A}}
{}
$$

\end{frame}

\begin{frame}
\frametitle{Model Theory for Classical Logic}

{\bf Boolean Algebra} $\langle \mathbb{B} , \false , \true , \&\& , || , ! \rangle$

Classical truth is a boolean value.

\begin{block}{Satisfaction}
\begin{align*}
\sat{A} &\dfn \eval{A} \equiv \true\\
\nsat{A} &\dfn \eval{A} \equiv \false
\end{align*}
\end{block}

\begin{block}{Evaluation}
\begin{align*}
\eval{p} &\dfn \sigma ~ p\\
\eval{A \land B} &\dfn \eval{A} ~\&\&~ \eval{B}\\
\eval{A \lor B} &\dfn \eval{A} ~||~ \eval{B}\\
\eval{A \imp B} &\dfn ! (\eval{A}) ~||~ \eval{B}\\
\eval{\neg A} &\dfn ! (\eval{A})
\end{align*}
\end{block}

\end{frame}

\begin{frame}
\frametitle{Model Theory for Intuitionistic Logic}

{\bf Kripke Model} $\langle \mathbb{C} , \leq , \varnothing , \Vdash \rangle$

Intuitionistic truth is constructive evidence, or a program.


\begin{block}{Forcing (intuitionistic satisfaction)}
\begin{align*}
\force{p} &\dfn \forcep{p}\\
\force{A \land B} &\dfn \force{A} \times \force{B}\\
\force{A \lor B} &\dfn \force{A} ~\uplus~ \force{B}\\
\force{A \imp B} &\dfn \Gamma \leq \Delta \arr \dforce{A} \arr \dforce{B}\\
\force{\neg A} &\dfn \Gamma \leq \Delta \arr \dforce{A} \arr \bot\\
\nforce{A} &\dfn \force{\neg A}
\end{align*}
\end{block}

\end{frame}

\begin{frame}
\frametitle{Classical vs Intuitionistic Model Theory}

Many more intuitionistic models than classical models
because intuitionistic implication and negation allow 
arbitrary intrinsically distinct functions.

$\Rightarrow$ Much bigger search space for an intuitionistic theorem prover!

\begin{block}{Evaluation}
\begin{align*}
\eval{A \imp B} &\dfn ! (\eval{A}) ~||~ \eval{B}\\
\eval{\neg A} &\dfn ! (\eval{A})
\end{align*}
\end{block}

\begin{block}{Forcing}
\begin{align*}
\force{A \imp B} &\dfn \Gamma \leq \Delta \arr \dforce{A} \arr \dforce{B}\\
\force{\neg A} &\dfn \Gamma \leq \Delta \arr \dforce{A} \arr \bot
\end{align*}
\end{block}


\end{frame}


\begin{frame}
\frametitle{Intuitionistic Tableau Calculus}

$$
\infer
  [\T \land]
  {S , \T A , \T B}
{
  {S , \T(A \land B)}
}
\qquad
\infer
  [\F \land]
  {S, \F A \| S, \F B}
{
  {S, \F(A \land B)}
}
$$

$$
\infer
  [\T \lor]
  {S , \T A \| S , \T B}
{
  {S , \T(A \lor B)}
}
\qquad
\infer
  [\F \lor]
  {S, \F A , \F B}
{
  {S, \F(A \lor B)}
}
$$

\end{frame}




\begin{frame}
\frametitle{References}
\framesubtitle{Classical is to Intuitionistic as Smullyan is to Heyting}

\end{frame}

\end{document}

