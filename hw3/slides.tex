\documentclass[mathserif]{beamer}
\usetheme{Warsaw}
\usepackage{proof}

\title{Tableau Theorem Prover\\ for Intuitionistic Propositional Logic}
\author{Larry Diehl}
\institute{Portland State University}
\date[Fall 2014]
{CS 510 - Mathematical Logic and Programming Languages}

\def\|{~|~}
\def\T{\textrm{T}}
\def\F{\textrm{F}}
\def\true{\textrm{true}}
\def\false{\textrm{false}}
\def\imp{\supset}
\def\dfn{~\Leftrightarrow~}
\def\arr{\Rightarrow}

\newcommand{\diff}[1]{\textcolor{red}{#1}}

\newcommand{\turn}[1]{\Gamma \vdash #1}
\newcommand{\cturn}[2]{\Gamma , #1 \vdash #2}

\newcommand{\eval}[1]{\sigma \triangleright #1}
\newcommand{\sat}[1]{\sigma \vDash #1}
\newcommand{\nsat}[1]{\sigma \nvDash #1}
\newcommand{\force}[1]{\Gamma \Vdash #1}
\newcommand{\dforce}[1]{\Delta \Vdash #1}
\newcommand{\forcep}[1]{\Gamma \Vdash^p #1}
\newcommand{\nforce}[1]{\Gamma \nVdash #1}

\def\mycalc{
$$
\infer
  [\T \land]
  {S , \T A , \T B}
{
  {S , \T(A \land B)}
}
\qquad
\infer
  [\F \land]
  {S, \F A \| S, \F B}
{
  {S, \F(A \land B)}
}
$$

$$
\infer
  [\T \lor]
  {S , \T A \| S , \T B}
{
  {S , \T(A \lor B)}
}
\qquad
\infer
  [\F \lor]
  {S, \F A , \F B}
{
  {S, \F(A \lor B)}
}
$$

$$
\infer
  [\T \! \imp]
  {S , \F A \| S , \T B}
{
  {S , \T(A \imp B)}
}
\qquad
\infer
  [\F \! \imp]
  {S_{\T}, \T A , \F B}
{
  {S, \F(A \imp B)}
}
$$

$$
\infer
  [\T \neg]
  {S , \F A}
{
  {S , \T(\neg A)}
}
\qquad
\infer
  [\F \neg]
  {S_{\T}, \T A}
{
  {S, \F(\neg A)}
}
$$
}

\begin{document}

\frame{\titlepage}

\begin{frame}
\frametitle{Motivation}
\framesubtitle{Tableau for Classical Logic}

\begin{itemize}
\item If $\neg A$ is contradictory in all paths, then $A \lor \neg A$
  lets us conclude $A$ is a {\it tautology}.
\item For {\it satisfiability}, running tableau on $A$ yield a
  (classical model) evaluation context $\sigma$.
\item Tableau seems awfully tied to classical logic, is intuitionistic
  tableau doomed!?
\end{itemize}



\end{frame}


\begin{frame}
\frametitle{Classical vs Intuitionistic Logic}

\begin{block}{Classical Logic}
\begin{itemize}
\item The {\it meaning} of a proposition is its truth value.
\item {\bf Satisfiability:} Does evaluating it yield true?
\item $A \lor \neg A$
\item $\neg\neg A \imp A$
\item $A \imp \neg\neg A$
\end{itemize}
\end{block}

\begin{block}{Intuitionistic Logic}
\begin{itemize}
\item The {\it meaning} of a proposition is its constructive content.
\item {\bf Satisfiability:} Can you write it as a program?
\item $A \imp \neg\neg A$
\end{itemize}
\end{block}

\end{frame}


\begin{frame}
\frametitle{Proof Theory for Intuitionistic Logic}

$$
\infer
  [\imp_I]
  {\turn{A \imp B}}
{
  {\cturn{A}{B}}
}
\qquad
\infer
  [\imp_E]
  {\turn{B}}
{
  {\turn{A \imp B}}
  &
  {\turn{A}}
}
$$

$$
\infer
  [\land_I]
  {\turn{A \land B}}
{
  {\turn{A}}
  &
  {\turn{B}}
}
\qquad
\infer
  [\land_{E_1}]
  {\turn{A}}
{
  {\turn{A \land B}}
}
\qquad
\infer
  [\land_{E_2}]
  {\turn{B}}
{
  {\turn{A \land B}}
}
$$

$$
\infer
  [\lor_{I_1}]
  {\turn{A \lor B}}
{
  {\turn{A}}
}
\qquad
\infer
  [\lor_{I_2}]
  {\turn{A \lor B}}
{
  {\turn{B}}
}
\qquad
\infer
  [\lor_E]
  {\turn{C}}
{
  {\cturn{A}{C}}
  &
  {\cturn{B}{C}}
}
$$

$$
\infer
  [\neg_I]
  {\turn{\neg A}}
{
  {\cturn{A}{\bot}}
}
\qquad
\infer
  [\neg_E]
  {\turn{\bot}}
{
  {\turn{A}}
  &
  {\turn{\neg A}}
}
$$

$$
\infer
  [\top_I]
  {\turn{\top}}
{
}
\qquad
\infer
  [\bot_E]
  {\turn{A}}
{
  {\turn{\bot}}
}
$$

\end{frame}

\begin{frame}
\frametitle{Proof Theory for Classical Logic}

\center{... intuitionistic rules plus ...}

$$
\infer
  [\neg\neg_I]
  {\turn{\neg\neg A}}
{
  {\turn{A}}
}
\qquad
\infer
  [\neg\neg_E]
  {\turn{A}}
{
  {\turn{\neg\neg A}}
}
$$

\center{...or...}

$$
\infer{\turn{A \lor \neg A}}
{}
$$

\end{frame}

\begin{frame}
\frametitle{Model Theory for Classical Logic}

{\bf Boolean Algebra} $\langle \mathbb{B} , \false , \true , \&\& , || , ! \rangle$

Classical truth is a boolean value.

\begin{block}{Satisfiability}
\begin{align*}
\sat{A} &\dfn \eval{A} \equiv \true\\
\nsat{A} &\dfn \eval{A} \equiv \false
\end{align*}
\end{block}

\begin{block}{Evaluation}
\begin{align*}
\eval{p} &\dfn \sigma ~ p\\
\eval{A \land B} &\dfn \eval{A} ~\&\&~ \eval{B}\\
\eval{A \lor B} &\dfn \eval{A} ~||~ \eval{B}\\
\eval{A \imp B} &\dfn ! (\eval{A}) ~||~ \eval{B}\\
\eval{\neg A} &\dfn ! (\eval{A})
\end{align*}
\end{block}

\end{frame}

\begin{frame}
\frametitle{Model Theory for Intuitionistic Logic}

{\bf Kripke Model} $\langle \mathbb{C} , \leq , \varnothing , \Vdash \rangle$

Intuitionistic truth is constructive evidence, or a program.


\begin{block}{Forcing (intuitionistic satisfiability)}
\begin{align*}
\force{p} &\dfn \forcep{p}\\
\force{A \land B} &\dfn \force{A} \times \force{B}\\
\force{A \lor B} &\dfn \force{A} ~\uplus~ \force{B}\\
\force{A \imp B} &\dfn \Gamma \leq \Delta \arr \dforce{A} \arr \dforce{B}\\
\force{\neg A} &\dfn \Gamma \leq \Delta \arr \dforce{A} \arr \bot\\
\nforce{A} &\dfn \force{\neg A}
\end{align*}
\end{block}

\end{frame}

\begin{frame}
\frametitle{Classical vs Intuitionistic Model Theory}

Many more intuitionistic models than classical models
because intuitionistic implication and negation allow 
arbitrary intrinsically distinct functions.

{\it Much bigger search space for an intuitionistic theorem prover!}

\begin{block}{Evaluation}
\begin{align*}
\eval{A \imp B} &\dfn ! (\eval{A}) ~||~ \eval{B}\\
\eval{\neg A} &\dfn ! (\eval{A})
\end{align*}
\end{block}

\begin{block}{Forcing}
\begin{align*}
\force{A \imp B} &\dfn \Gamma \leq \Delta \arr \dforce{A} \arr \dforce{B}\\
\force{\neg A} &\dfn \Gamma \leq \Delta \arr \dforce{A} \arr \bot
\end{align*}
\end{block}


\end{frame}

\begin{frame}
\frametitle{Classical Tableau Calculus}

$$
\infer
  [\T \land]
  {S , \T A , \T B}
{
  {S , \T(A \land B)}
}
\qquad
\infer
  [\F \land]
  {S, \F A \| S, \F B}
{
  {S, \F(A \land B)}
}
$$

$$
\infer
  [\T \lor]
  {S , \T A \| S , \T B}
{
  {S , \T(A \lor B)}
}
\qquad
\infer
  [\F \lor]
  {S, \F A , \F B}
{
  {S, \F(A \lor B)}
}
$$

$$
\infer
  [\T \! \imp]
  {S , \F A \| S , \T B}
{
  {S , \T(A \imp B)}
}
\qquad
\infer
  [\F \! \imp]
  {S_{\T}, \T A , \F B}
{
  {S, \F(A \imp B)}
}
$$

$$
\infer
  [\T \neg]
  {S , \F A}
{
  {S , \T(\neg A)}
}
\qquad
\infer
  [\F \neg]
  {S_{\T}, \T A}
{
  {S, \F(\neg A)}
}
$$

\end{frame}


\begin{frame}[label=calculus]
\frametitle{Intuitionistic Tableau Calculus}

$$
\diff{S_{\T} \dfn \{ \T A \| \T A \in S \}}
$$

$$
\infer
  [\T \land]
  {S , \T A , \T B}
{
  {S , \T(A \land B)}
}
\qquad
\infer
  [\F \land]
  {S, \F A \| S, \F B}
{
  {S, \F(A \land B)}
}
$$

$$
\infer
  [\T \lor]
  {S , \T A \| S , \T B}
{
  {S , \T(A \lor B)}
}
\qquad
\infer
  [\F \lor]
  {S, \F A , \F B}
{
  {S, \F(A \lor B)}
}
$$

$$
\infer
  [\T \! \imp]
  {S , \F A \| S , \T B}
{
  {S , \T(A \imp B)}
}
\qquad
\infer
  [\F \! \imp]
  {\diff{S_{\T}}, \T A , \F B}
{
  {S, \F(A \imp B)}
}
$$

$$
\infer
  [\T \neg]
  {S , \F A}
{
  {S , \T(\neg A)}
}
\qquad
\infer
  [\F \neg]
  {\diff{S_{\T}}, \T A}
{
  {S, \F(\neg A)}
}
$$

\end{frame}


\begin{frame}
\frametitle{Classical Tableau Interpretation}

Gradually build an evaluation context $\sigma$ for $A$ (such that $\sat{A}$),
until tableau is finished or the model is contradictory.

\begin{block}{Judgments}
\begin{itemize}
\item $\T A$ means $A$ is true in the model.
\item $\F A$ means $A$ is false in the model.
\end{itemize}
\end{block}

\begin{block}{Inference Rules}
If the premise is true, then the conclusion is true.
\end{block}

\end{frame}

\begin{frame}
\frametitle{Intuitionistic Tableau Interpretation}

Gradually build a ``proof'' of $A$ (an ``element'' of $\force{A}$),
until tableau is finished or the model is contradictory.

\begin{block}{Judgments}
\begin{itemize}
\item $\T A$ means we have a proof of $A$.
\item $\F A$ means $A$ we do not (yet) have a proof of $A$.
\end{itemize}
\end{block}

\begin{block}{Inference Rules}
\begin{itemize}
\item If the premise is true, then the conclusion \diff{may} be true.
\item The conclusion is logically consistent with the premise.
\end{itemize}
\end{block}

\end{frame}

\againframe{calculus}

\begin{frame}
\frametitle{Closed Example $A \imp A$}

\begin{columns}[T]
\begin{column}[T]{4cm}
\begin{align*}
[\{ \F(A \diff{\imp} A) \}],\\
[\{ \T A, \F A \}].
\end{align*}
\end{column}

\begin{column}[T]{7cm}
\mycalc
\end{column}
\end{columns}

\end{frame}

\begin{frame}
\frametitle{Closed Example $A \imp (A \land A)$}

\begin{columns}[T]
\begin{column}[T]{4cm}
\begin{align*}
[\{ \F(A \diff{\imp} (A \land A)) \}],\\
[\{ \T A, \F(A \diff{\land} A)) \}],\\
[\{ \T A, \F A \}, \{ \T A, \F A \}].
\end{align*}
\end{column}

\begin{column}[T]{7cm}
\mycalc
\end{column}
\end{columns}

\end{frame}

\begin{frame}
\frametitle{Open Example $A \lor \neg A$}

\begin{columns}[T]
\begin{column}[T]{4cm}
\begin{align*}
[\{ \F(A \diff{\lor} \neg A) \}],\\
[\{ \F A, \F(\diff{\neg} A) \}],\\
[\{ \T A \}].
\end{align*}
\end{column}

\begin{column}[T]{7cm}
\mycalc
\end{column}
\end{columns}

\end{frame}

\begin{frame}
\frametitle{Closed Example $A \imp (A \imp B) \imp B$}

\begin{columns}[T]
\begin{column}[T]{4cm}
\begin{align*}
[\{ \F(A \diff{\imp} (A \imp B) \imp B) \}],\\
[\{ \T A , \F((A \imp B) \diff{\imp} B) \}],\\
[\{ \T A, \T(A \diff{\imp} B) , \F B \}],\\
[\{ \T A, \F A, \F B \}, \{ \T A, \T B , \F B \}].
\end{align*}
\end{column}

\begin{column}[T]{7cm}
\mycalc
\end{column}
\end{columns}

\end{frame}


\begin{frame}
\frametitle{Classical vs Intuitionistic Tableau Search}

When looking for a closed tableau:

\begin{block}{Classical}
You can prioritize {\bf any} rule to apply to $S$ to shrink the search space.
\end{block}

\begin{block}{Intuitionistic}
You must try applying {\bf all} rules to $S$,
but can still prioritize some and backtrack if they fail.
\end{block}

\end{frame}

\begin{frame}
\frametitle{``Open'' Example $\neg A \imp \neg A$}

\begin{columns}[T]
\begin{column}[T]{4cm}
\begin{align*}
[\{ \F(\neg A \diff{\imp} \neg A) \}],\\
[\{ \T(\diff{\neg} A), \F(\neg A) \}],\\
[\{ \F A, \F(\diff{\neg} A) \}],\\
[\{ \T A \}].
\end{align*}
\end{column}

\begin{column}[T]{7cm}
\mycalc
\end{column}
\end{columns}

\end{frame}

\begin{frame}
\frametitle{Closed Example $\neg A \imp \neg A$}

\begin{columns}[T]
\begin{column}[T]{4cm}
\begin{align*}
[\{ \F(\neg A \diff{\imp} \neg A) \}],\\
[\{ \T(\neg A), \F(\diff{\neg} A) \}],\\
[\{ \T(\diff{\neg} A), \T A \}],\\
[\{ \F A, \T A \}].
\end{align*}
\end{column}

\begin{column}[T]{7cm}
\mycalc
\end{column}
\end{columns}

\end{frame}

\begin{frame}
\frametitle{Using Classical vs Intuitionistic Tableau}

\begin{block}{Classical}
To show that $A$ is true:
\begin{enumerate}
\item Assume that $A$ is false.
\item Build a tableau for $\neg A$.
\item If some sub-proposition is true and false, $A$ must be true.
\end{enumerate}
\end{block}

\begin{block}{Intuitionistic}
To show that $A$ is provable:
\begin{enumerate}
\item Assume that $A$ has not been proven.
\item Build a tableau for $\neg A$.
\item If some sub-proposition is provable and not yet proven,
it must be impossible that $A$ has not been proven.
\end{enumerate}
\end{block}

\end{frame}

\begin{frame}
\frametitle{Classical vs Intuitionistic Tableau Soundness}

\begin{block}{Classical}
{\bf Have} a model $\sigma$ from the tableau {\it conclusion}, so
{\bf check} that $\sat{A}$.
\end{block}

\begin{block}{Intuitionistic}
{\bf Have} a tableau {\it derivation} of $A$, so
{\bf construct} an element of $\force{A}$.
\end{block}

\end{frame}

\begin{frame}
\frametitle{Intuitionistic Tableau Soundness}

\begin{block}{Theorem}
{\bf Have} a tableau {\it derivation} of $A$, so
{\bf construct} an element of $\force{A}$.
\end{block}

\begin{block}{Fitting's Proof}
By showing the contrapositive.\\
Sadly, $(\neg B \imp \neg A) \nRightarrow (A \imp B)$ intuitionistically.
\end{block}

\end{frame}

\begin{frame}
\frametitle{References}
\framesubtitle{Classical is to Intuitionistic as Smullyan is to Fitting}

\begin{block}{Classical Tableau Book}
{\it First Order Logic} - Smullyan'68
\end{block}

\begin{block}{Intuitionistic Tableau Book}
{\it Intuitionistic Logic: Model Theory and Forcing} - Fitting'69
\end{block}

\begin{block}{Intuitionistic Tableau Optimization Papers}
\begin{itemize}
\item {\it An O(n log n)-Space Decision Procedure for Intuitionistic Propositional Logic} - Hudelmaier'93
\item {\it A Tableau Decision Procedure for Propositional Intuitionistic Logic} - Avellone et. al.'06
\end{itemize}
\end{block}

\end{frame}





\end{document}

