\documentclass[mathserif]{beamer}
\usetheme{Warsaw}
\usepackage{proof}

\title{Tableau Theorem Prover\\ for Intuitionistic Propositional Logic}
\author{Larry Diehl}
\institute{Portland State University}
\date[Fall 2014]
{CS 510 - Mathematical Logic and Programming Languages}

\def\imp{\supset}
\newcommand{\turn}[1]{\Gamma \vdash #1}
\newcommand{\cturn}[2]{\Gamma , #1 \vdash #2}

\begin{document}

\frame{\titlepage}

\begin{frame}
\frametitle{Proof Theory for Classical Logic}

$$
\infer
  [\imp_I]
  {\turn{A \imp B}}
{
  {\cturn{A}{B}}
}
\qquad
\infer
  [\imp_E]
  {\turn{B}}
{
  {\turn{A \imp B}}
  &
  {\turn{A}}
}
$$

$$
\infer
  [\land_I]
  {\turn{A \land B}}
{
  {\turn{A}}
  &
  {\turn{B}}
}
\qquad
\infer
  [\land_{E_1}]
  {\turn{A}}
{
  {\turn{A \land B}}
}
\qquad
\infer
  [\land_{E_2}]
  {\turn{B}}
{
  {\turn{A \land B}}
}
$$

$$
\infer
  [\lor_{I_1}]
  {\turn{A \lor B}}
{
  {\turn{A}}
}
\qquad
\infer
  [\lor_{I_2}]
  {\turn{A \lor B}}
{
  {\turn{B}}
}
\qquad
\infer
  [\lor_E]
  {\turn{C}}
{
  {\cturn{A}{C}}
  &
  {\cturn{B}{C}}
}
$$

$$
\infer
  [\neg_I]
  {\turn{\neg A}}
{
  {\cturn{A}{\bot}}
}
\qquad
\infer
  [\neg_E]
  {\turn{\bot}}
{
  {\turn{A}}
  &
  {\turn{\neg A}}
}
$$

$$
\infer
  [\top_I]
  {\turn{\top}}
{
}
\qquad
\infer
  [\bot_E]
  {\turn{C}}
{
  {\turn{\bot}}
}
$$

\end{frame}

\begin{frame}
\frametitle{This is the second slide}
\framesubtitle{A bit more information about this}

\end{frame}

\begin{frame}
\frametitle{Reference Smullyan vs Heyting books}

\end{frame}

\end{document}

